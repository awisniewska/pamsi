\documentclass{mwart} 
\usepackage[polish]{babel} 
\usepackage[utf8]{inputenc} 
\usepackage{polski} 
\usepackage[T1]{fontenc} 
\usepackage{graphicx}

\usepackage[margin=1cm]{geometry}




\frenchspacing 

\usepackage{indentfirst} 
\title{\Huge{Sprawozdanie z laboratorium nr 8\\
Graf nieskierowany z wagą }}
\author{Agnieszka Wiśniewska, nr albumu: 200466}

\begin{document}

\maketitle

\section{Wprowadzenie}
Graf jest strukturą danych do reprezentacji obiektów, między którymi występują różne zależności. Podstawowymi elementami grafu są wierzchołki połączone ze sobą krawędziami. Graf nieskierowany (obecny program) to taki, w którym krawędzie można przebywać w obie strony (kolejność wierzchołków w parze definiującej krawędź nie jest istotna). Wagi są to wartości związane z krawędziami.

\newpage
\section{Podsumowanie i wnioski}
\begin{itemize}
\item h
\end{itemize}




\end{document}